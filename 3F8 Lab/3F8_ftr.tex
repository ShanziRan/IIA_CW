%% LyX 2.2.4 created this file.  For more info, see http://www.lyx.org/.
%% Do not edit unless you really know what you are doing.
\documentclass[english]{article}
\usepackage{lmodern}
\usepackage[T1]{fontenc}
\usepackage[latin9]{inputenc}
\usepackage{geometry}
\geometry{verbose,tmargin=3cm,bmargin=3cm,lmargin=2.5cm,rmargin=2.5cm}
\usepackage{textcomp}
\usepackage{graphicx}

\makeatletter

%%%%%%%%%%%%%%%%%%%%%%%%%%%%%% LyX specific LaTeX commands.
%% Because html converters don't know tabularnewline
\providecommand{\tabularnewline}{\\}

\makeatother

\usepackage{babel}
\begin{document}

\title{3F8: Inference\\
 Full Technical Report}

\author{Author's Name}
\maketitle
\begin{abstract}
This is the abstract. 

Try for 1-2 sentences on each of: motive (what it\textquoteright s
about), method (what was

done), key results and conclusions (the main outcomes).

\textbullet{} Don\textquoteright t exceed 3 sentences on any one.

\textbullet{} Write this last, when you know what it should say!
\end{abstract}

\section{Introduction}
\begin{enumerate}
\item What is the problem and why is it interesting?
\item What novel follow-up will the rest of your report present?
\end{enumerate}

\section{Exercise a)}

{[} Describe how the Laplace approximation works in general and how
to apply it to the Logistic classifier. Provide a justification of
each step. {]}

\section{Exercise b)}

{[} Describe the new gradient form, the python code and any specific
implementation details {]}
\begin{verbatim}
#
# Python code to be included
#
\end{verbatim}

\section{Exercise c)}

{[} Include plots in Figure \ref{fig:predictive_distributions} and
describe how the results differ from each other {]}

\begin{figure}
\begin{centering}
\includegraphics[width=0.3\paperwidth]{place_holder_figure}\hspace{1cm}\includegraphics[width=0.3\paperwidth]{place_holder_figure}
\par\end{centering}
\caption{Plots showing data and contour lines for the predictive distribution
generated by the Laplace approximation (left) and the MAP solution
(right).\label{fig:predictive_distributions} }
\end{figure}


\section{Exercise d)}

{[} Include results in Tables \ref{tab:ll_MAP}, \ref{tab:ll_Laplace},
\ref{tab:conf_MAP} and \ref{tab:conf_Laplace} and explain the results
obtained and any findings {]}

\begin{table}
\centering{}%
\begin{minipage}[t]{0.5\textwidth}%
\begin{center}
\begin{tabular}{c|c}
\textbf{Avg. Train ll} & \textbf{Avg. Test ll}\tabularnewline
\hline 
- & -\tabularnewline
\hline 
\end{tabular}\caption{Log-likelihoods for MAP solution.\label{tab:ll_MAP}}
\par\end{center}%
\end{minipage}%
\begin{minipage}[t]{0.5\textwidth}%
\begin{center}
\begin{tabular}{c|c}
\textbf{Avg. Train ll} & \textbf{Avg. Test ll}\tabularnewline
\hline 
- & -\tabularnewline
\hline 
\end{tabular}\caption{Log-likelihoods for Laplace approximation.\label{tab:ll_Laplace}}
\par\end{center}%
\end{minipage}
\end{table}

\begin{table}
\centering{}%
\begin{minipage}[t]{0.5\textwidth}%
\begin{center}
\begin{tabular}{cc|c|c}
 & \multicolumn{1}{c}{} & \multicolumn{1}{c}{$\hat{y}$} & \tabularnewline
 &  & 0 & 1\tabularnewline
\cline{2-4} 
$y$ & 0 & - & -\tabularnewline
\cline{2-4} 
 & 1 & - & -\tabularnewline
\cline{2-4} 
\end{tabular} 
\par\end{center}
\caption{Conf. matrix for for MAP solution.\label{tab:conf_MAP}}
%
\end{minipage}%
\begin{minipage}[t]{0.5\textwidth}%
\begin{center}
\begin{tabular}{cc|c|c}
 & \multicolumn{1}{c}{} & \multicolumn{1}{c}{$\hat{y}$} & \tabularnewline
 &  & 0 & 1\tabularnewline
\cline{2-4} 
$y$ & 0 & - & -\tabularnewline
\cline{2-4} 
 & 1 & - & -\tabularnewline
\cline{2-4} 
\end{tabular} 
\par\end{center}
\caption{Conf. matrix for Laplace approximation.\label{tab:conf_Laplace}}
%
\end{minipage}
\end{table}


\section{Exercise e)}

{[} describe your grid search approach, the python code, the grid
points chosen, the heat map plot from Figure \ref{fig:heat_map_plot}
and the best hyper-parameter values obtained via grid search {]}
\begin{verbatim}
#
# Python code to be included
#
\end{verbatim}
\begin{figure}
\begin{centering}
\includegraphics[width=0.3\paperwidth]{place_holder_figure}
\par\end{centering}
\caption{Heat map plot of the the approximation of the model evidence obtained
in the grid search.\label{fig:heat_map_plot} }
\end{figure}


\section{Exercise f)}

{[} Describe the visualisation of the predictions in Figure \ref{fig:prediction_visualisation_after_tuning}
and the results in Tables \ref{tab:average_ll_after_tuning} and \ref{tab:confusion_after_tuning}.
How do they compare to the ones obtained in previous exercises? {]}

\begin{figure}
\begin{centering}
\includegraphics[width=0.3\paperwidth]{place_holder_figure}
\par\end{centering}
\caption{Visualisation of the contours of the class predictive probabilities
for Laplace approximation after hyper-parameter tuning by maximising
the model evidence.\label{fig:prediction_visualisation_after_tuning} }
\end{figure}

\begin{table}
\centering{}%
\begin{minipage}[t]{0.4\columnwidth}%
\begin{center}
\vspace{-0.2cm}%
\begin{tabular}{c|c}
\textbf{Avg. Train ll} & \textbf{Avg. Test ll}\tabularnewline
\hline 
- & -\tabularnewline
\hline 
\end{tabular} 
\par\end{center}
\caption{Average training and test log-likelihoods for Laplace approximation
after hyper-parameter tuning by maximising the model evidence.\label{tab:average_ll_after_tuning}}
%
\end{minipage}\hspace{2cm}%
\begin{minipage}[t]{0.4\columnwidth}%
\begin{center}
\begin{tabular}{cc|c|c}
 & \multicolumn{1}{c}{} & \multicolumn{1}{c}{$\hat{y}$} & \tabularnewline
 &  & 0 & 1\tabularnewline
\cline{2-4} 
$y$ & 0 & - & -\tabularnewline
\cline{2-4} 
 & 1 & - & -\tabularnewline
\cline{2-4} 
\end{tabular} 
\par\end{center}
\caption{Confusion matrix for Laplace approximation after hyper-parameter tuning
by maximising the model evidence.\label{tab:confusion_after_tuning}}
%
\end{minipage}
\end{table}


\section{Conclusions}
\begin{enumerate}
\item Draw together the most important results and their consequences.
\item List any reservations or limitations.
\end{enumerate}

\end{document}
