%% LyX 2.2.4 created this file.  For more info, see http://www.lyx.org/.
%% Do not edit unless you really know what you are doing.
\documentclass[english]{article}
\usepackage{lmodern}
\usepackage[T1]{fontenc}
\usepackage[latin9]{inputenc}
\usepackage{geometry}
\geometry{verbose,tmargin=3cm,bmargin=3cm,lmargin=2.5cm,rmargin=2.5cm}
\usepackage{textcomp}
\usepackage{amstext, amsmath}
\usepackage{graphicx}

\makeatletter

%%%%%%%%%%%%%%%%%%%%%%%%%%%%%% LyX specific LaTeX commands.
%% Because html converters don't know tabularnewline
\providecommand{\tabularnewline}{\\}

\makeatother

\usepackage{babel}
\begin{document}

\title{3F8: Inference\\
 Short Lab Report}

\author{Author's Name}
\maketitle
\begin{abstract}
This is the abstract. 

Try for 1-2 sentences on each of: motive (what it\textquoteright s
about), method (what was

done), key results and conclusions (the main outcomes).

\textbullet{} Don\textquoteright t exceed 3 sentences on any one.

\textbullet{} Write this last, when you know what it should say!
\end{abstract}

\section{Introduction}
\begin{enumerate}
\item What is the problem and why is it interesting?
\item What novel follow-up will the rest of your report present?
\end{enumerate}

\section{Exercise a)\label{sec:Exercise-a}}

In this exercise we have to consider the logistic classication model
(aka logistic regression) and derive the gradients of the log-likelihood
given a vector of binary labels $\mathbf{y}$ and a matrix of input
features $\mathbf{X}$. The gradient of the log-likelihood can be
writen as
\begin{align*}
    \frac{\partial\mathcal{L}(\beta)}{\partial\beta} &= \frac{\partial}{\partial\beta} \prod_{n=1}^{N} log\{\sigma(\beta^T \tilde{x}^{(n)})^{y^{(n)}} (1-\sigma(\beta^T \tilde{x}^{(n)}))^{1-y^{(n)}}\} \\
    &= \sum_{n=1}^{N} \{y^{(n)}\frac{\partial}{\partial\beta}log\{\sigma(\beta^T \tilde{x}^{(n)})\} + (1-y^{(n)})\frac{\partial}{\partial\beta}log\{1-\sigma(\beta^T \tilde{x}^{(n)})\}\}
\end{align*}

Using derivative results (substituting $ z = \beta^T \tilde{x}^{(n)} $):
\begin{align*}
    \frac{\partial}{\partial z} \log \sigma(z) &= \frac{1}{\sigma(z)}\sigma(z) (1 - \sigma(z)) = 1 - \sigma(z) \\
    \frac{\partial}{\partial z} \log (1 - \sigma(z)) &= \frac{1}{1 - \sigma(z)} (-\sigma(z) (1 - \sigma(z))) = -\sigma(z) \\
\end{align*}

We can then write:
\begin{align*}
    \frac{\partial\mathcal{L}(\beta)}{\partial\beta} &= \sum_{n=1}^{N} \{ y^{(n)} (1 - \sigma(\beta^T \tilde{x}^{(n)})) \tilde{x}^{(n)} - (1 - y^{(n)}) \sigma(\beta^T \tilde{x}^{(n)}) \tilde{x}^{(n)} \} \\  
    &= \sum_{n=1}^{N} \{y^{(n)} - \sigma(\beta^T \tilde{x}^{(n)}) \} \tilde{x}^{(n)}
\end{align*}

\section{Exercise b)}

In this exercise we are asked to write pseudocode to estimate the
parameters $\beta$ using gradient ascent of the log-likelihood. Our
code should be vectorised. The pseudocode to estimate the parameters
$\beta$ is shown below:
\begin{verbatim}
Function estimate_parameters:

   Input:  feature matrix X, labels y
   Output: vector of coefficients b
	
    max_iter = 1000 # maximum number of iterations set to avoid infinite training
    eta = 0.01 # learning rate (choice explained below)
    b = zeros(D) # initialise coefficients vector with dimention D
    for iter in range(max_iter):
        grad_old = X.T @ (y - sigmoid(X @ b)) # calculate old gradient
        b_new = b + eta * grad_old # gradient ascent
        if norm(b_new - b) == 0: # check whether has reached local maximum
            break
        b = b_new
   return b
\end{verbatim}
The learning rate parameter $\eta$ is chosen to avoid both divergence due to too large $\eta$ and slow convergence due to too small $\eta$.
Therefore, it is here set to the widely accepeted default value of $0.01$.
\section{Exercise c)}

In this exercise we visualise the dataset in the two-dimensional input
space displaying each datapoint's class label. The dataset is visualised
in Figure \ref{fig:data_visualisation}. By analising Figure \ref{fig:data_visualisation}
we conclude that a linear classifier...

\begin{figure}
\begin{centering}
% \includegraphics[width=0.3\paperwidth]{place_holder_figure}
\par\end{centering}
\caption{Visualisation of the data.\label{fig:data_visualisation} }
\end{figure}


\section{Exercise d)}

In this exercise we split the data randomly into training and test
sets with 800 and 200 data points, respectively. The pseudocode from
exercise a) is transformed into python code as follows:
\begin{verbatim}
#
# Python code to be included
#
\end{verbatim}
We then train the classifier using this code. We fixed the learning
rate parameter to be $\eta=...$ . The average log-likelihood on the
training and test sets as the optimisation proceeds are shown in Figure
\ref{fig:learning_curves}. By looking at these plots we conclude
that ...

Figure \ref{fig:learning_curves} displays the visualisation of the
contours of the class predictive probabilities on top of the data.
This figure shows that...

\begin{figure}
\begin{centering}
% \includegraphics[width=0.3\paperwidth]{place_holder_figure}\hspace{1cm}\includegraphics[width=0.3\paperwidth]{place_holder_figure}
\par\end{centering}
\caption{Learning curves showing the average log-likelihood on the training
(left) and test (right) datasets.\label{fig:learning_curves} }
\end{figure}

\begin{figure}
\begin{centering}
% \includegraphics[width=0.3\paperwidth]{place_holder_figure}
\par\end{centering}
\caption{Visualisation of the contours of the class predictive probabilities.\label{fig:data_visualisation-1} }
\end{figure}


\section{Exercise e)}

The final average training and test log-likelihoods are shown in Table
\ref{tab:average_ll}. These results indicate that... The 2x2 confusion
matrices on the and test set is shown in Table \ref{tab:confusion_test}.
By analising this table, we conclude that...

\begin{table}
\centering{}%
\begin{minipage}[t]{0.49\columnwidth}%
\begin{center}
\begin{tabular}{c|c}
\textbf{Avg. Train ll} & \textbf{Avg. Test ll}\tabularnewline
\hline 
- & -\tabularnewline
\hline 
\end{tabular} 
\par\end{center}
\caption{Average training and test log-likelihoods.\label{tab:average_ll}}
%
\end{minipage}%
\begin{minipage}[t]{0.49\columnwidth}%
\begin{center}
\begin{tabular}{cc|c|c}
 & \multicolumn{1}{c}{} & \multicolumn{1}{c}{$\hat{y}$} & \tabularnewline
 &  & 0 & 1\tabularnewline
\cline{2-4} 
$y$ & 0 & - & -\tabularnewline
\cline{2-4} 
 & 1 & - & -\tabularnewline
\cline{2-4} 
\end{tabular} 
\par\end{center}
\caption{Confusion matrix on the test set.\label{tab:confusion_test}}
%
\end{minipage}
\end{table}


\section{Exercise f)}

We now expand the inputs through a set of Gaussian radial basis functions
centred on the training datapoints. We consider widths $l=\{0.01,0.1,1\}$
for the basis functions. We fix the learning rate parameter to be
$\eta=\{...,...,...\}$ for each $l=\{0.01,0.1,1\}$, respectively.
Figure \ref{fig:contours_l} displays the visualisation of the contours
of the resulting class predictive probabilities on top of the data
for each choice of $l=\{0.01,0.1,1\}$.
\begin{figure}
\begin{centering}
% \includegraphics[width=0.32\textwidth]{place_holder_figure}\hspace{0.15cm}\includegraphics[width=0.32\textwidth]{place_holder_figure}\hspace{0.15cm}\includegraphics[width=0.32\textwidth]{place_holder_figure}
\par\end{centering}
\caption{Visualisation of the contours of the class predictive probabilities
for $l=0.01$ (left), $l=0.1$ (middle), $l=1$ (right).\label{fig:contours_l} }
\end{figure}


\section{Exercise g)}

The final final training and test log-likelihoods per datapoint obtained
for each setting of $l=\{0.01,0.1,1\}$ are shown in tables \ref{tab:avg_ll_l_001},
\ref{tab:avg_ll_l_01} and \ref{tab:avg_ll_l_1}. These results indicate
that... The 2 \texttimes{} 2 confusion matrices for the three models
trained with $l=\{0.01,0.1,1\}$ are show in tables \ref{tab:conf_l_001},
\ref{tab:conf_l_01} and \ref{tab:conf_l_1}. After analysing these
matrices, we can say that... When we compare these results to those
obtained using the original inputs we conclude that...

\begin{table}
\centering{}%
\begin{minipage}[t]{0.3\textwidth}%
\begin{center}
\begin{tabular}{c|c}
\textbf{Avg. Train ll} & \textbf{Avg. Test ll}\tabularnewline
\hline 
- & -\tabularnewline
\hline 
\end{tabular}\caption{Results for $l=0.01$\label{tab:avg_ll_l_001}}
\par\end{center}%
\end{minipage}\hspace{0.5cm}%
\begin{minipage}[t]{0.3\textwidth}%
\begin{center}
\begin{tabular}{c|c}
\textbf{Avg. Train ll} & \textbf{Avg. Test ll}\tabularnewline
\hline 
- & -\tabularnewline
\hline 
\end{tabular}\caption{Results for $l=0.1$\label{tab:avg_ll_l_01}}
\par\end{center}%
\end{minipage}\hspace{0.5cm}%
\begin{minipage}[t]{0.3\textwidth}%
\begin{center}
\begin{tabular}{c|c}
\textbf{Avg. Train ll} & \textbf{Avg. Test ll}\tabularnewline
\hline 
- & -\tabularnewline
\hline 
\end{tabular}\caption{Results for $l=1$\label{tab:avg_ll_l_1}}
\par\end{center}%
\end{minipage}
\end{table}

\begin{table}
\centering{}%
\begin{minipage}[t]{0.33\textwidth}%
\begin{center}
\begin{tabular}{cc|c|c}
 & \multicolumn{1}{c}{} & \multicolumn{1}{c}{$\hat{y}$} & \tabularnewline
 &  & 0 & 1\tabularnewline
\cline{2-4} 
$y$ & 0 & - & -\tabularnewline
\cline{2-4} 
 & 1 & - & -\tabularnewline
\cline{2-4} 
\end{tabular} 
\par\end{center}
\caption{Conf. matrix $l=0.01$.\label{tab:conf_l_001}}
%
\end{minipage}%
\begin{minipage}[t]{0.33\textwidth}%
\begin{center}
\begin{tabular}{cc|c|c}
 & \multicolumn{1}{c}{} & \multicolumn{1}{c}{$\hat{y}$} & \tabularnewline
 &  & 0 & 1\tabularnewline
\cline{2-4} 
$y$ & 0 & - & -\tabularnewline
\cline{2-4} 
 & 1 & - & -\tabularnewline
\cline{2-4} 
\end{tabular} 
\par\end{center}
\caption{Conf. matrix $l=0.1$.\label{tab:conf_l_01}}
%
\end{minipage}%
\begin{minipage}[t]{0.33\textwidth}%
\begin{center}
\begin{tabular}{cc|c|c}
 & \multicolumn{1}{c}{} & \multicolumn{1}{c}{$\hat{y}$} & \tabularnewline
 &  & 0 & 1\tabularnewline
\cline{2-4} 
$y$ & 0 & - & -\tabularnewline
\cline{2-4} 
 & 1 & - & -\tabularnewline
\cline{2-4} 
\end{tabular} 
\par\end{center}
\caption{Conf. matrix $l=1$.\label{tab:conf_l_1}}
%
\end{minipage}
\end{table}


\section{Conclusions}
\begin{enumerate}
\item Draw together the most important results and their consequences.
\item List any reservations or limitations.
\end{enumerate}

\end{document}
